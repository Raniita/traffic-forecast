\documentclass[a4paper, oneside, 12pt]{book}
%\documentclass[12pt]{article}
\usepackage[spanish, english, es-tabla]{babel}
\usepackage[utf8]{inputenc}
\usepackage[left = 2cm, right = 2cm, bottom = 2cm, top = 3cm]{geometry}
\usepackage{amsmath, amssymb}
\usepackage{graphicx}
\usepackage{hyperref}
\usepackage{listings}
\usepackage{courier}

\usepackage[dvipsnames]{xcolor}

% Portada Oficial
\usepackage[export]{adjustbox}

\renewcommand{\lstlistingname}{Ejemplo}
\renewcommand{\lstlistlistingname}{Listado de ejemplos}

% https://tex.stackexchange.com/questions/60209/how-to-add-an-extra-level-of-sections-with-headings-below-subsubsection
%\newcommand{\subsubsubsection}[1]{\paragraph{#1}\mbox{}\\}
%\setcounter{secnumdepth}{4}
%\setcounter{tocdepth}{4}

% Configuracion PDF metadata
\hypersetup{
	pdftitle= {Diseño y desarrollo de un microservicio para la monitorización y precicción de tráfico en red},
	pdfauthor = {Enrique Fernández Sánchez},
	pdfsubject = {Gestión de datos de monitorización y predicción de tráfico en red},
	pdfkeywords = {Microservicio, Monitorización, Predicción, Python, FastAPI, Network Forecast}
}

% Configuracion colores hiperenlaces
\hypersetup{
	colorlinks=true,
	linkcolor=black,
	%filecolor=magenta,      
	urlcolor=cyan,
}

% Configure lstlisting
\definecolor{codegreen}{rgb}{0,0.6,0}
\definecolor{codegray}{rgb}{0.5,0.5,0.5}
\definecolor{codepurple}{rgb}{0.58,0,0.82}
\definecolor{backcolour}{rgb}{0.95,0.95,0.92}

\lstdefinestyle{mystyle}{
	backgroundcolor=\color{backcolour},   
	commentstyle=\color{codegreen},
	keywordstyle=\color{magenta},
	numberstyle=\tiny\color{codegray},
	stringstyle=\color{codepurple},
	basicstyle=\ttfamily\footnotesize,
	breakatwhitespace=false,         
	breaklines=true,                 
	captionpos=t,                    
	keepspaces=true,                 
	numbers=left,                    
	numbersep=5pt,                  
	showspaces=false,                
	showstringspaces=false,
	showtabs=false,                  
	tabsize=2
}

\lstset{style=mystyle}

% Configuracion encabezados y pies de pagina
\usepackage{fancyhdr}
\fancyhf{}
\lhead[\leftmark]{\small TFM: Enrique Fernández Sánchez}
\rhead[Nombre Autor]{\rightmark}
\cfoot[\thepage]{}
\cfoot[]{\thepage}
\renewcommand{\headrulewidth}{0.5pt}
\renewcommand{\footrulewidth}{0pt}
\fancypagestyle{plain}{
	\fancyhf{}
	\fancyhead[L]{\small TFM: Enrique Fernández Sánchez}
	\fancyfoot[C]{\thepage}
	%\renewcommand{\headrulewidth}{0pt}		% Sirve para eliminar linea
	%\renewcommand{\footrulewidth}{0pt}		% Sirve para eliminar linea
}
\pagestyle{fancy}

\begin{document}
	\selectlanguage{spanish}
	
	%% PORTADA UPCT
	\thispagestyle{empty}
	
	\newgeometry{left=0.01cm, bottom=0.01cm, top=0.01cm}
	
	\begin{minipage}{0.2\textwidth}
		\includegraphics[height=\textheight, left]{img/banda_etsit_90.png}
	\end{minipage}
	\centerline{\begin{minipage}[t][6cm][b]{0.5\textwidth}
			\title{\textbf{Diseño y desarrollo de un microservicio para la gestión de información de monitorización y predicciones de tráfico en red}} 
			
			%\date{1 enero 2023}
			
			\maketitle
			
			\vspace{0.5cm}
			\hspace{1.5cm} \textbf{TRABAJO FIN DE MÁSTER} \\
			
			\vspace{0.5cm}
			\hspace{1.15cm} Máster Universitario en Ingeniería de \\   \vspace{-0.5cm}
			\hspace{3cm}Telecomunicación \\
			
			\vspace{2cm}
			\textbf{Autor:} \author{Enrique Fernández Sánchez} \\
			\textbf{Tutor:} Pablo Pavón Mariño
	\end{minipage}}
	
	\restoregeometry
	%% END PORTADA UPCT	
	
	\pagebreak
	
	\tableofcontents
	
	\pagebreak
	
	\addcontentsline{toc}{section}{Índice de figuras}
	\listoffigures
	
	\pagebreak
	
	\addcontentsline{toc}{section}{Listado de ejemplos}
	\lstlistoflistings
	
	\pagebreak
	
	\chapter{Introducción}
	
	\section{Contexto del trabajo}
	
	\section{Motivación}
	
	\section{Descripción Global}
	
	\section{Objetivos}
	
	\section{Resumen capítulos de la memoria}
	
	\pagebreak
	
	\chapter{Tecnologías empleadas}
	
	\section{Arquitectura y microservicios}
	
	\section{Bases de datos}
	
	\section{Modelo de predicción}
	
	\section{Lenguajes de programación y frameworks}
	
	\section{Tecnologías utilizadas en un despliegue en producción}
	
	\pagebreak
	
	\chapter{Diseño e implementación del sistema}
	
	\section{Descripción REST API}
	
	\section{Estructura de la aplicación}
	
	\section{Modelos de datos}
	
	\section{Endpoints}
	
	\section{Implementación del sistema}
	
	\pagebreak
	
	\chapter{Pruebas y validación del sistema}
	
	\pagebreak
	
	\chapter{Conclusiones}
	
	\section{Propuestas futuras}
	
	\pagebreak
	
	\chapter{Bibliografía}
	
	\pagebreak
	
	\chapter*{Anexos}
	\addcontentsline{toc}{chapter}{Anexos}
	
	\section*{Anexo I. Generación dataset sintético}
	\addcontentsline{toc}{section}{Anexo I. Generación dataset sintético}
	
\end{document}