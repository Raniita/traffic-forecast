\documentclass[aspectratio=169,xcolor=dvipsnames]{beamer}
% https://github.com/PM25/SimplePlus-BeamerTheme
\usetheme{SimplePlus}

\usepackage{hyperref}
\usepackage{graphicx} 
\usepackage{booktabs}
\usepackage{courier}

\usepackage[spanish]{babel}

% Portada

\title{\large Diseño y desarrollo de un microservicio para la gestión de información de monitorización y predicciones de tráfico en red}

\author{\small \textit{Autor}: Enrique Fernández Sánchez \\
\textit{Tutor}: Pablo Pavón Mariño}

\institute[UPCT]{\\ Universidad Politécnica de Cartagena (UPCT) \\ \vspace{5px}
	\includegraphics[scale=0.2]{img/escudo_upct.png}
}

\date{\today}

\usepackage{tikz}
\logo{ 
	\begin{tikzpicture}[overlay,remember picture, inner sep=0pt,outer sep=0pt]
		\node[yshift=-20px,left=0.2cm] at (current page.31){
			\includegraphics[width=3cm]{img/etsit.png}
		};
	\end{tikzpicture}
}

\begin{document}
	% 1
	\begin{frame}
		\titlepage
	\end{frame}

	% 2
	\begin{frame}{Índice}
		\tableofcontents
	\end{frame}

    % --------------------
    \section{Introducción}
    
    \begin{frame}{Introducción}
    
        \begin{itemize}
            \item \textit{Abstract}: Aplicación que permite almacenar muestras de monitorización de tráfico en red, y a su vez, generar predicciones futuras del tráfico de red, en función de la información almacenada.
        \end{itemize}
    
        \begin{block}{Objetivos del proyecto}
            \begin{itemize}
                \item Diseñar una \textbf{aplicación} siguiendo la metodología de \textbf{microservicios}.
                
                \item Investigar herramientas de \textbf{predicción de series temporales}.
                
                \item Investigar \textbf{opciones de almacenamiento} para \textbf{muestras temporales}.
                
                \item Utilizar \textbf{herramientas de documentación} que permitan conocer la estructura de la aplicación.\\
                
            \end{itemize}
        \end{block}
    \end{frame}

	% ---------------------
	
	\section{Tecnologías empleadas}

	\begin{frame}{Microservicios}
		\begin{exampleblock}{Definición de microservicio}
			Sistemas que cumplen las siguientes premisas: 
			
			\begin{itemize}
				\item Sistemas pequeños, independientes y poco ``acoplados''. 
				\item Código fuente separado entre los diferentes servicios, no necesariamente mismo lenguaje.
				\item Los servicios se comunican entre sí utilizando APIs.
				\item Cada sistema es independiente, y responsable de su persistencia de datos.
			\end{itemize}
		\end{exampleblock}
		
		\begin{alertblock}{Importancia de utilizar microservicios}
			\begin{itemize}
				\item Permite que otras aplicaciones más complejas puedan implementar el servicio propuesto.
			\end{itemize}
		\end{alertblock}
	\end{frame}

	% ---------------------
	
	\begin{frame}{\texttt{API} \small (\textit{Application Programing Interface})}
		\begin{itemize}
			\item Una API permite a dos \textbf{componentes comunicarse} entre sí \textbf{mediante} una serie de \textbf{reglas}.
			
			\item Supone un ``\textbf{contrato}'' en el que \textbf{se establecen las solicitudes} y respuestas \textbf{esperadas en la comunicación}.
		\end{itemize}
	
		\begin{exampleblock}{Tipos de API}
			Dependiendo de la implementación, distinguimos entre cuatro tipos:
			
			\begin{itemize}
				\item \textbf{SOAP}. Protocolo tradicional, usa mensajes XML (HTTP solo transporte). Ambos interlocutores deben conocer la estructura de los objetos. Poco flexible
				\item \textbf{RPC}. Basado en llamadas a procedimientos remotos. 
				\item \textbf{WebSocket}. Solicitud moderna, usa objetos JSON y un canal bidireccional de comunicación.
				\item \textbf{REST}. Solución más popular y flexible. Usa los métodos HTTP. \textbf{Elegimos este tipo}.
			\end{itemize}
		\end{exampleblock}
	\end{frame}
	
	% ---------------------
	
	\begin{frame}{Bases de datos}
		
		\begin{itemize}
			\item En función del tipo de dato a almacenar, se distinguen dos bases de datos dentro de la aplicación:
		\end{itemize}
		
		\begin{columns}
			\begin{column}{0.5\textwidth}
				\begin{block}{Tipo relacional (\texttt{PostgreSQL})}
					\begin{itemize}
						\item Se almacenan datos que puedan tener una relación entre ellos. Por ejemplo: información de una red a monitorizar.
						
						\item Se organizan los datos en una serie de ``relaciones'', y estas se almacenan en una o más ``tablas''.
						
						\item Cada tabla dispone de una serie de columnas. La información se agrega en forma de filas a la tabla.
					\end{itemize}
				\end{block}
			\end{column}
			
			
			\begin{column}{0.5\textwidth}
				\begin{block}{Tipo serie temporal (\texttt{InfluxDB})}
					\begin{itemize}
						\item Necesario para almacenar datos en forma de serie temporal de manera eficiente.
						
						\item En comparación con el tipo relacional, la fila de datos esta identificada por un valor temporal.
						
						\item Algunas ventajas:
						\begin{itemize}
							\item Útil para almacenar datos de telemetría.
							\item Datos comprimidos automáticamente.
							%\item Permite realizar consultas de manera sencilla.							
						\end{itemize}
					\end{itemize}
				\end{block}
			\end{column}
		\end{columns}
	\end{frame}
	
	% ---------------------
	
	\begin{frame}{Lenguaje \& frameworks}
		\begin{columns}
			\begin{column}{0.5\textwidth}
				\begin{exampleblock}{\texttt{Python}}
					Lenguaje de programación orientado a objetos, interpretado y de alto nivel. Muy popular en los siguientes ambitos:
					
					\begin{itemize}
						\item Aplicaciones web.
						
						\item Data Science
						
						\item Inteligencia Artificial
					\end{itemize}
					
					\begin{figure}[h!]
						\begin{center}
							\includegraphics[width=0.7\textwidth]{img/python_logo.png}
							%\caption{asd}
							%\label{img: microservice architecture}
						\end{center}
					\end{figure}
				\end{exampleblock}
			\end{column}
		
			\begin{column}{0.5\textwidth}
				\begin{exampleblock}{\texttt{FastAPI}}
					Framework moderno y rápido para construir APIs. Características:
					
					\begin{itemize}
						\item Rápido: rendimiento equivalente a otros lenguajes (NodeJS o Go).
						\item Intuitivo: soporta autocompletado. 
						\item Robusto: herramienta Swagger/ReDoc automática.
					\end{itemize}
					
					\begin{figure}[h!]
						\begin{center}
							\includegraphics[width=0.8\textwidth]{img/fastapi_logo.png}
							%\caption{asd}
							%\label{img: microservice architecture}
						\end{center}
					\end{figure}
				\end{exampleblock}
				
				
			\end{column}
		\end{columns}
	\end{frame}

	\begin{frame}{Lenguaje \& frameworks}
		\begin{exampleblock}{\texttt{Prophet}}
			Framework del lenguaje de programación Python, desarrollado por Meta (Facebook). Agrupa una serie de procedimientos que permiten realizar predicciones en un dataset temporal. Características:
			
			\begin{itemize}
				\item Permite encontrar y tener en cuenta efectos no lineales (tendencias diarias, semanales, mensuales...).
				\item Permite predecir datos en días vacacionales.
				\item Basado en inferencia estadística, más eficiente que si utilizáramos técnicas de \textit{Machine Learning}.
			\end{itemize}
			
			\begin{figure}[h!]
				\begin{center}
					\includegraphics[width=0.4\textwidth]{img/prophet_logo.png}
					%\caption{asd}
					%\label{img: microservice architecture}
				\end{center}
			\end{figure}
		\end{exampleblock}
	\end{frame}
	
	% ---------------------
	
	\begin{frame}{Despliegue en producción}
		Herramientas utilizadas para el despliegue del sistema en un entorno de producción:
		
		\begin{itemize}
			\item \texttt{Docker}. Contenedores.
			\item \texttt{docker-compose}. Despliegue automatizado de contenedores \texttt{Docker}.
			\item \texttt{Traefik}. Enrutador que permite conectar dominios con servicios \texttt{Docker}.
		\end{itemize}
	
		\begin{figure}[h!]
			\begin{center}
				\includegraphics[width=0.95\textwidth]{diag/produccion_tfm.png}
			\end{center}
		\end{figure}
	\end{frame}
	
	% ---------------------
	
	\section{Implementación del sistema}
	
	\begin{frame}{Descripción API REST (I)}
		\begin{columns}
			\begin{column}{0.5\textwidth}
				Para la aplicación se implementan los diferentes agentes utilizando dos metodologías: \\
				
				\vspace{12px}
				
				\begin{enumerate}
					\item \textbf{CRUD} 
					\begin{itemize}
						\item Aquellas que siguen la definición \texttt{Create}, \texttt{Read}, \texttt{Update} \& \texttt{Delete}.
						
						\item \textit{Ejemplo}: redes o interfaces.
					\end{itemize}
				
					\item \textbf{No CRUD}
					\begin{itemize}
						\item Aquellas que no tienen por qué seguir las reglas CRUD.
						
						\item \textit{Ejemplo}: ejecutar predicción de red.
					\end{itemize}
				\end{enumerate}
			\end{column}
		
			\begin{column}{0.5\textwidth}
				\begin{alertblock}{Agentes}
					Se identifican los diferentes ``agentes'' presentes en el sistema:
					
					\begin{itemize}
						\item \textbf{Redes} (\texttt{networks}), corresponde con una red que contiene interfaces a monitorizar.
						
						\item \textbf{Interfaces} (\texttt{interfaces}), corresponde con las interfaces de red que queremos monitorizar.
						
						\item \textbf{Muestras de monitorización} (\texttt{samples}). Valor de tráfico asociado a una interfaz en un intervalo determinado.
					\end{itemize}
				\end{alertblock}
			\end{column}
		\end{columns}
	\end{frame}
	
	% ---------------------
	
	\begin{frame}{Descripción API REST (II)}
		
		\textbf{Esquema de la aplicación implementada}
		
		\begin{figure}[h!]
			\begin{center}
				\includegraphics[width=0.75\textwidth]{diag/traffic_forecast_schema.png}
			\end{center}
		\end{figure}
	\end{frame}
	
	% ---------------------
	
	\begin{frame}{Implementación (I)}
		\begin{exampleblock}{\texttt{Factory Pattern}}
			Metodología que permite estructurar un proyecto software de modo que permita:
			\begin{itemize}
				\item Añadir funcionalidades de manera sencilla.
				\item Permitir el crecimiento ordenado de la aplicación.
				\item Ordenar los archivos del proyecto dependiendo de la funcionalidad.
			\end{itemize}
		\end{exampleblock}
	
		\begin{alertblock}{\texttt{ORM} \& \texttt{migrations}}
			\begin{itemize}
				\item \texttt{ORM} (\textit{Object-relational Mappers}). Abstracción de alto nivel que permite definir modelos de datos SQL con el lenguaje Python.
				
				\item \texttt{migrations}. Permite tener un control de versiones dentro de los modelos de datos de la aplicación. Facilita el despliegue de la base de datos, en lo que a estructura de tablas se refiere.	
		\end{itemize}
		\end{alertblock}
		
	\end{frame}

	\begin{frame}{Implementación (II)}
		\begin{columns}
			\begin{column}{0.6\textwidth}
				\textbf{Estructura de ficheros del proyecto}
				
				\begin{itemize}
					\item \texttt{migrations}. Contiene las versiones de la DB.
					
					\item \texttt{src}
					
					\begin{itemize}
						\item \texttt{crud}. Funcionalidad de agentes CRUD.
						
						\item \texttt{database}. Contiene modelos de datos SQL.
						
						\item \texttt{noncrud} Funcionalidad agentes no CRUD.
						
						\item \texttt{routes}. Rutas de la aplicación.
						
						\item \texttt{schemas}. Esquema de datos.
						
						\item \texttt{utils}. Utilidades, \texttt{wrapper InfluxDB}.
						
						\item \texttt{config.py}. Configuración app.
						
						\item \texttt{main.py}. Archivo principal.
					\end{itemize}
				
					\item \texttt{Dockerfile}. Instrucciones para contenedor Docker.
					
					\item \texttt{requirements.txt}. Módulos Python utilizados.
				\end{itemize}
			\end{column}
		
			\begin{column}{0.4\textwidth}
				\begin{figure}[h!]
					\begin{center}
						\includegraphics[width=0.65\textwidth]{img/folder_structure.jpg}
					\end{center}
				\end{figure}
			\end{column}
		\end{columns}
	\end{frame}
	
	% ---------------------
	
	\begin{frame}{OpenAPI. \texttt{Swagger} \& \texttt{ReDoc}}
		\begin{itemize}
			\item FastAPI permite \textbf{generar de manera automática} una serie de rutas que sirven como \textbf{documentación del proyecto}.
			
			\item En estas herramientas se muestran al usuario la información de: 
			
			\begin{itemize}
				\item Rutas de la aplicación
				
				\item Parámetros de entrada y de salida para una petición.
				
				\item Modelos de datos para cada una de las rutas.
				
				\item Posibles códigos HTTP de respuesta para las peticiones.
			\end{itemize}
		
			\item \textbf{Permite} al usuario r\textbf{ealizar peticiones al servidor}, y ver su resultado, desde la misma herramienta.
		\end{itemize}
		
		\begin{block}{Acceso a las herramientas de documentación}
			\begin{itemize}
				\item \textbf{Swagger}: \href{https://tfm-api.ranii.pro:8443/docs}{\texttt{https://tfm-api.ranii.pro:8443/docs}}
				
				\item \textbf{ReDoc}: \href{https://tfm-api.ranii.pro:8443/redoc}{\texttt{https://tfm-api.ranii.pro:8443/redoc}}
			\end{itemize}
		\end{block}
	\end{frame}
	
	% ---------------------
	
	\begin{frame}{Modelos de datos. \texttt{SQL}}
		
		\begin{itemize}
			\item Definimos los datos y su \textbf{representación dentro} de nuestra \textbf{aplicación}.
			
			\item \textbf{En función} del \textbf{tipo de dato}, almacenaremos en \textbf{base de datos tipo relacional} (\texttt{SQL}) o \textbf{base de datos tipo serie temporal} (\texttt{InfluxDB}).
		\end{itemize}

		\vspace{11px}

		\textbf{Tablas utilizadas en la base de datos SQL}

		\begin{columns}
			\begin{column}{0.5\textwidth}
				\begin{table}[h!]
					\centering
					\begin{tabular}{|l|}
						\hline
						\multicolumn{1}{|c|}{\textit{\textbf{Networks}}} \\ \hline
						\texttt{id\_network}: Int, Public Key, Unique                 \\ \hline
						\texttt{name}: String                                     \\ \hline
						\texttt{description}: String                              \\ \hline
						\texttt{ip\_red}: String                                  \\ \hline
						\texttt{influx\_net}: String                              \\ \hline
					\end{tabular}
					%\caption{Modelo de datos para las redes a monitorizar. Equivale con la tabla \textit{networks}.}
					%\label{tab: modelo sql networks}
				\end{table}
			\end{column}
		
			\begin{column}{0.5\textwidth}
				\begin{table}[h!]
					\centering
					\begin{tabular}{|l|}
						\hline
						\multicolumn{1}{|c|}{\textit{\textbf{Interfaces}}} \\ \hline
						\texttt{id\_interface}: Int, Public Key, Unique             \\ \hline
						\texttt{name}: String                                       \\ \hline
						\texttt{description}: String                                \\ \hline
						\texttt{influx\_if\_rx}: String                             \\ \hline
						\texttt{influx\_if\_tx}: String                             \\ \hline
						\texttt{network}: Int, Foreign Key                          \\ \hline
					\end{tabular}
					%\caption{Modelo de datos para las interfaces a monitorizar. Equivale con la tabla \textit{interfaces}.}
					%\label{tab: modelo sql interfaces}
				\end{table}
			\end{column}
		\end{columns}
	\end{frame}

	\begin{frame}{Modelos de datos. \texttt{InfluxDB}}
		\begin{itemize}
			\item En \texttt{InfluxDB} almacenamos las muestras de tráfico en red.
		\end{itemize}
	
		\vspace{11px}
	
		\textbf{Configuración InfluxDB para el sistema}
	
		\begin{table}[]
			\centering
			\begin{tabular}{|c|c|}
				\hline
				\textbf{InfluxDB}    & \textbf{Descripción}                                                                                                                            \\ \hline
				\texttt{measurement} & \begin{tabular}[c]{@{}c@{}}Red a monitorizar, \\ valor almacenado en \texttt{Networks::influx\_net}\end{tabular}                                         \\ \hline
				\texttt{fields}      & Nombre del valor a monitorizar (\textit{link\_count})                                                                                                    \\ \hline
				\texttt{tags}        & \begin{tabular}[c]{@{}c@{}}Solo disponemos un \texttt{tag}, llamado \texttt{interface}, \\ contiene la información del identificador de una interfaz\end{tabular} \\ \hline
				\texttt{points}      & Corresponde con el valor numérico del campo \texttt{field}.                                                                                              \\ \hline
			\end{tabular}
			%\caption{}
			%\label{tab:my-table}
		\end{table}
	
	\end{frame}
	
	% ---------------------
	
	\begin{frame}{Predicción de tráfico de red (I)}
		\begin{itemize}
			\item El sistema es \textbf{capaz} de \textbf{ejecutar predicciones} de tráfico de red, en función de las \textbf{muestras de monitorización} previamente \textbf{almacenadas en el sistema}.
		\end{itemize}
	
		\begin{exampleblock}{Pasos para realizar una predicción de tráfico}
			\begin{enumerate}
				\item Creamos una red a monitorizar. \texttt{POST} a \texttt{/networks}.
				
				\item Creamos una interfaz a monitorizar. \texttt{POST} a \texttt{/networks/<net>/interfaces}
				
				\item Importamos datos de monitorización.  Dos maneras: 
				\begin{itemize}
					\item \texttt{POST} a ruta: \texttt{/samples/<net>/import\_topology}
					
					\item \texttt{POST} a ruta: \texttt{/samples/<net>/import\_interface/<if>}
				\end{itemize}
			
				\item Ejecutamos predicción de tráfico en red. Configuramos la predicción y lanzamos la petición \texttt{POST} a \texttt{/forecast}.
			\end{enumerate}
		\end{exampleblock}
	\end{frame}
	
	% ---------------------
	
	\begin{frame}{Predicción de tráfico de red (II)}
		\begin{columns}
			\begin{column}{0.4\textwidth}
				\begin{figure}[h!]
					\begin{center}
						\includegraphics[width=0.9\textwidth]{img/parameters_post_forecast.png}
					\end{center}
				\end{figure}
			\end{column}
		
			\begin{column}{0.6\textwidth}
				\textbf{Parámetros para la ejecución de la predicción de tráfico}
				
				\begin{itemize}
					\item \texttt{field}. Campo para elegir si RX o TX.
					
					\item \texttt{days}. Número de días a predecir.
					
					\item \texttt{options}. Opciones que modifican la flexibilidad de la predicción:
					\begin{itemize}
						\item \texttt{holidays\_region}
						\item \texttt{flexibility\_trend}
						\item \texttt{flexibility\_season}
						\item \texttt{flexibility\_holidays}
					\end{itemize}
				\end{itemize}
			\end{column}
		\end{columns}
	\end{frame}
	
	% ---------------------
	
	\begin{frame}{Resumen rutas HTTP (I)}
		\begin{columns}
			\begin{column}{0.5\textwidth}
				\begin{block}{\textbf{CRUD}: \texttt{networks} (\texttt{/networks})}
					\begin{itemize}
						\item \textbf{Información} de \textbf{todas} las redes: \\ \textit{GET} - \texttt{/networks}
						
						\item \textbf{Crear} una red: \\ \textit{POST} - \texttt{/networks}
						
						\item \textbf{Información} de \textbf{una} red: \\ \textit{GET} - \texttt{/networks/<net\_id>}
						
						\item \textbf{Eliminar} una red: \\ \textit{DELETE} - \texttt{/networks/<net\_id>}
						
						\item \textbf{Actualizar} una red: \\ \textit{PATCH} - \texttt{/networks/<net\_id>}
					\end{itemize}
				\end{block}
			\end{column}
		
			\begin{column}{0.5\textwidth}
				\begin{block}{\textbf{CRUD}: \texttt{interfaces} (\texttt{/networks/<id1>/interfaces})}
					\begin{itemize}
						\item \textbf{Información} de \textbf{todas} las interfaces: \\ \textit{GET} - \texttt{\small /networks/id/interfaces}
						
						\item \textbf{Crear} una interfaz: \\ \textit{POST} - \texttt{\small /networks/id/interfaces}
						
						\item \textbf{Información} de \textbf{una} interfaz: \\ \textit{GET} - \texttt{/../interfaces/<id2>}
						
						\item \textbf{Eliminar} una interfaz: \\ \textit{DELETE} - \texttt{/../interfaces/<id2>}
						
						\item \textbf{Actualizar} una interfaz: \\ \textit{PATCH} - \texttt{/../interfaces/<id2>}
					\end{itemize}
				\end{block}
			\end{column}
		
		\end{columns}
	\end{frame}

	\begin{frame}{Resumen rutas HTTP (II)}
		\begin{columns}
			\begin{column}{0.5\textwidth}
				\begin{exampleblock}{\texttt{Samples}}
					\begin{itemize}
						\item \textbf{Importar} datos de topología con \textbf{más de una interfaz}: \\ \textit{\small POST} - \texttt{\small /samples/id/import\_topology}
						
						\item \textbf{Importar} datos de \textbf{una interfaz}: \\ \textit{POST} - \texttt{\small /samples/id/import\_interface/id}
					\end{itemize}
				\end{exampleblock}
			\end{column}
		
			\begin{column}{0.5\textwidth}
				\begin{exampleblock}{\texttt{Query Samples}}
					\begin{itemize}
						\item \textbf{Consultar} \textbf{datos} de monitorización \textbf{almacenados}: \\ \textit{GET} - \texttt{\small /query/}
					\end{itemize}
				\end{exampleblock}
			
				\begin{exampleblock}{\texttt{Forecast}}
					\begin{itemize}
						\item \textbf{Ejecutar} \textbf{predicción} de tráfico en red: \\ \textit{POST} - \texttt{\small /forecast/}
					\end{itemize}
				\end{exampleblock}
			\end{column}
		\end{columns}
	\end{frame}
	
	% ---------------------
	
	\section{Validación del sistema}
	
	\begin{frame}{Validación del sistema (I)}
		Para demostrar el funcionamiento del sistema, se realiza una \textbf{batería de pruebas} de las funcionalidades implementadas. Dichas pruebas son:
		
		\begin{enumerate}
			\item \textbf{Crear una red} a monitorizar.
			
			\item \textbf{Crear una interfaz} dentro de una red a monitorizar.
			
			\item \textbf{Cargar muestras} en interfaz de red.
			
			\item Cargar una topología de red sobre una red a monitorizar.
			
			\item \textbf{Consultar datos} de monitorización.
			
			\item \textbf{Ejecutar una predicción} de un año.
		\end{enumerate}
	
	\vspace{12px}
	
	Estas pruebas corresponden con las \textbf{pruebas unitarias} del sistema.
		
	\end{frame}

	\begin{frame}{Validación del sistema (II)}
		\begin{columns}
			\begin{column}{0.52\textwidth}
				\begin{block}{Demo! (\textit{2 minutos})}
					Realizamos una validación del sistema.
					
					\vspace{12px}
					
					\href{https://tfm-api.ranii.pro:8443/docs}{\small \texttt{https://tfm-api.ranii.pro:8443/docs}}
					
					\vspace{12px}
					 
					\begin{itemize}
						\item \textit{Objetivo}: Completar una predicción de un año. 
						\item \textit{Requisito}: Asumir sistema sin datos almacenados.
						\item \textit{Resultado}: Una predicción similar a la de la figura.
					\end{itemize}
				\end{block}
			\end{column}
		
			\begin{column}{0.5\textwidth}
				\begin{figure}[h!]
					\begin{center}
						\includegraphics[width=1\textwidth]{img/graph_forecast_2.png}
					\end{center}
				\end{figure}
			\end{column}
		\end{columns}
	\end{frame}
	
	% ---------------------
	
	\section{Conclusiones}
	
	\begin{frame}{Conclusiones}
		\begin{itemize}
			\item Alcanzados todos los objetivos propuestos.
			
			\item Se ha desarrollado una \textbf{microservicio completo}, permitiendo ser implementado por otras aplicaciones.
			
			\item El sistema es \textbf{capaz} de \textbf{almacenar} muestras de monitorización, además de poder \textbf{generar predicciones} de tráfico en red en la escala temporal que el usuario solicite.
		\end{itemize}
	
		\begin{exampleblock}{Propuestas futuras}
			\begin{itemize}
				\item Permitir la \textbf{importación} de \textbf{datos de monitorización} de \textbf{herramientas especificas} de planificación de red.
				
				\item Añadir \textbf{funcionalidad de SSE} (\textit{Server Side Event}) para \textbf{tareas} que requieran un \textbf{largo tiempo de ejecución}.
				
				\item Extender la funcionalidad de las \textbf{predicciones}, permitiendo hacer selección \textbf{más selectiva}, o añadir \textbf{filtrados extra}.
			\end{itemize}
		\end{exampleblock}
	\end{frame}
	
	% ---------------------
	
	\section{Bibliografía}
	
	\begin{frame}{Bibliografía}
		\begin{itemize}
		    \item La contenida en la memoria del proyecto: páginas 53 - 54
		\end{itemize}
	\end{frame}
	
	\begin{frame}
	    \begin{center}
	        \vspace{30px}
	        \Large \textbf{Muchas gracias por su atención} \\
	        \vspace{30px}
	        \large ¿Preguntas? \\
	        \vspace{60px}
	        
	        Enlace a la aplicación: \\
	        \href{https://tfm-api.ranii.pro:8443/docs}{\texttt{https://tfm-api.ranii.pro:8443/}}
	    \end{center}
	\end{frame}
	
\end{document}