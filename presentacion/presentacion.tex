\documentclass[aspectratio=169,xcolor=dvipsnames]{beamer}
% https://github.com/PM25/SimplePlus-BeamerTheme
\usetheme{SimplePlus}

\usepackage{hyperref}
\usepackage{graphicx} 
\usepackage{booktabs}
\usepackage{courier}

\usepackage[spanish]{babel}

% Portada

\title{\large Diseño y desarrollo de un microservicio para la gestión de información de monitorización y predicciones de tráfico en red}

\author{\small \textit{Autor}: Enrique Fernández Sánchez \\
\textit{Tutor}: Pablo Pavón Mariño}

\institute[UPCT]{\\ Universidad Politécnica de Cartagena (UPCT) \\ \vspace{5px}
	\includegraphics[scale=0.2]{img/escudo_upct.png}
}

\date{\today}

\usepackage{tikz}
\logo{ 
	\begin{tikzpicture}[overlay,remember picture, inner sep=0pt,outer sep=0pt]
		\node[yshift=-20px,left=0.2cm] at (current page.31){
			\includegraphics[width=3cm]{img/etsit.png}
		};
	\end{tikzpicture}
}

\begin{document}
	% 1
	\begin{frame}
		\titlepage
	\end{frame}

	% 2
	\begin{frame}{Índice}
		\tableofcontents
	\end{frame}

    % --------------------
    \section{Introducción}
    
    \begin{frame}{Introducción}
    
        \begin{itemize}
            \item \textit{Abstract}: Aplicación que permite almacenar muestras de monitorización de tráfico en red, y a su vez, generar predicciones futuras del tráfico de red, en función de la información almacenada.
        \end{itemize}
    
        \begin{block}{Objetivos del proyecto}
            \begin{itemize}
                \item Diseñar una aplicación siguiendo la metodología de microservicio.
                
                \item Investigar herramientas de predicción de series temporales.
                
                \item Investigar opciones de almacenamiento para muestras temporales.
                
                \item Utilizar herramientas de documentación que permitan conocer la estructura de la aplicación.\\
                
            \end{itemize}
        \end{block}
    \end{frame}

	% ---------------------
	
	\section{Tecnologías empleadas}

	\begin{frame}{Microservicios \& API}
		\begin{exampleblock}{Definición microservicio}
			content...
		\end{exampleblock}
		
		\begin{alertblock}{API}
			content...
		\end{alertblock}
	\end{frame}

	% ---------------------
	
	\begin{frame}{Bases de datos}
		\begin{columns}
			\begin{column}{0.5\textwidth}
				\begin{block}{Tipo relacional}
					asdasdasda
				\end{block}
			\end{column}
			
			
			\begin{column}{0.5\textwidth}
				\begin{block}{Tipo serie temporal}
					content...
				\end{block}
			\end{column}
		\end{columns}
	\end{frame}
	
	% ---------------------
	
	\begin{frame}{Lenguaje de programación \& frameworks}
		\begin{columns}
			\begin{column}{0.5\textwidth}
				\begin{exampleblock}{Python}
					Lenguaje de programación orientado a objetos, interpretado y de alto nivel. Muy popular en los siguientes ambitos:
					
					\begin{itemize}
						\item Aplicaciones web.
						
						\item Data Science
						
						\item Inteligencia Artificial
					\end{itemize}
					
					\begin{figure}[h!]
						\begin{center}
							\includegraphics[width=0.7\textwidth]{img/python_logo.png}
							\caption{asd}
							%\label{img: microservice architecture}
						\end{center}
					\end{figure}
				\end{exampleblock}
			\end{column}
		
			\begin{column}{0.5\textwidth}
				\begin{exampleblock}{FastAPI}
					Framework moderno y rápido para construir APIs. Rápido, intuitivo y robusto, además de basado en estándares de la industria.
					
					\begin{itemize}
						\item Rápido
						\item Intuitivo
						\item Robusto
					\end{itemize}
					
					
					\begin{figure}[h!]
						\begin{center}
							\includegraphics[width=0.7\textwidth]{img/fastapi_logo.png}
							\caption{asd}
							%\label{img: microservice architecture}
						\end{center}
					\end{figure}
				\end{exampleblock}
				
				
			\end{column}
		\end{columns}
	\end{frame}

	\begin{frame}{Lenguaje de programación \& frameworks}
		\begin{exampleblock}{Prophet}
			Framework del lenguaje de programación Python, desarrollado por Meta (Facebook)
			
			\begin{figure}[h!]
				\begin{center}
					\includegraphics[width=0.5\textwidth]{img/prophet_logo.png}
					\caption{asd}
					%\label{img: microservice architecture}
				\end{center}
			\end{figure}
		\end{exampleblock}
	\end{frame}
	
	% ---------------------
	
	\begin{frame}{Despliegue en producción}
		content...
	\end{frame}
	
	% ---------------------
	
	\section{Implementación del sistema}
	
	\begin{frame}{Descripción API REST (I)}
		content...
	\end{frame}
	
	% ---------------------
	
	\begin{frame}{Descripción API REST (II)}
		content...
	\end{frame}
	
	% ---------------------
	
	\begin{frame}{Estructura (I)}
		content...
	\end{frame}

	\begin{frame}{Estructura (II)}
		content...
	\end{frame}
	
	% ---------------------
	
	\begin{frame}{OpenAPI. Swagger}
		content...
	\end{frame}
	
	% ---------------------
	
	\begin{frame}{Modelos de datos. SQL}
		
		\begin{itemize}
			\item asdasd
		\end{itemize}

		\begin{columns}
			\begin{column}{0.5\textwidth}
				\begin{table}[h!]
					\centering
					\begin{tabular}{|l|}
						\hline
						\multicolumn{1}{|c|}{\textit{\textbf{Networks}}} \\ \hline
						\texttt{id\_network}: Int, Public Key, Unique                 \\ \hline
						\texttt{name}: String                                     \\ \hline
						\texttt{description}: String                              \\ \hline
						\texttt{ip\_red}: String                                  \\ \hline
						\texttt{influx\_net}: String                              \\ \hline
					\end{tabular}
					%\caption{Modelo de datos para las redes a monitorizar. Equivale con la tabla \textit{networks}.}
					%\label{tab: modelo sql networks}
				\end{table}
			\end{column}
		
			\begin{column}{0.5\textwidth}
				\begin{table}[h!]
					\centering
					\begin{tabular}{|l|}
						\hline
						\multicolumn{1}{|c|}{\textit{\textbf{Interfaces}}} \\ \hline
						\texttt{id\_interface}: Int, Public Key, Unique             \\ \hline
						\texttt{name}: String                                       \\ \hline
						\texttt{description}: String                                \\ \hline
						\texttt{influx\_if\_rx}: String                             \\ \hline
						\texttt{influx\_if\_tx}: String                             \\ \hline
						\texttt{network}: Int, Foreign Key                          \\ \hline
					\end{tabular}
					%\caption{Modelo de datos para las interfaces a monitorizar. Equivale con la tabla \textit{interfaces}.}
					%\label{tab: modelo sql interfaces}
				\end{table}
			\end{column}
		\end{columns}
	\end{frame}

	\begin{frame}{Modelos de datos. InfluxDB}
		content...
	\end{frame}
	
	% ---------------------
	
	\begin{frame}{Predicción de tráfico de red}
		content...
	\end{frame}
	
	% ---------------------
	
	\begin{frame}{Resumen rutas HTTP (I)}
		\begin{columns}
			\begin{column}{0.5\textwidth}
				\begin{block}{\textbf{CRUD}: \texttt{networks} (\texttt{/networks})}
					\begin{itemize}
						\item \textbf{Información} de \textbf{todas} las redes: \\ \textit{GET} - \texttt{/networks}
						
						\item \textbf{Crear} una red: \\ \textit{POST} - \texttt{/networks}
						
						\item \textbf{Información} de \textbf{una} red: \\ \textit{GET} - \texttt{/networks/<net\_id>}
						
						\item \textbf{Eliminar} una red: \\ \textit{DELETE} - \texttt{/networks/<net\_id>}
						
						\item \textbf{Actualizar} una red: \\ \textit{PATCH} - \texttt{/networks/<net\_id>}
					\end{itemize}
				\end{block}
			\end{column}
		
			\begin{column}{0.5\textwidth}
				\begin{block}{\textbf{CRUD}: \texttt{interfaces} (\texttt{/networks/<id1>/interfaces})}
					\begin{itemize}
						\item \textbf{Información} de \textbf{todas} las interfaces: \\ \textit{GET} - \texttt{\small /networks/id/interfaces}
						
						\item \textbf{Crear} una interfaz: \\ \textit{POST} - \texttt{\small /networks/id/interfaces}
						
						\item \textbf{Información} de \textbf{una} interfaz: \\ \textit{GET} - \texttt{/../interfaces/<id2>}
						
						\item \textbf{Eliminar} una interfaz: \\ \textit{DELETE} - \texttt{/../interfaces/<id2>}
						
						\item \textbf{Actualizar} una interfaz: \\ \textit{PATCH} - \texttt{/../interfaces/<id2>}
					\end{itemize}
				\end{block}
			\end{column}
		
		\end{columns}
	\end{frame}

	\begin{frame}{Resumen rutas HTTP (II)}
		content...
	\end{frame}
	
	% ---------------------
	
	\section{Validación del sistema}
	
	\begin{frame}{Validación del sistema (I)}
		content...
		
		\begin{exampleblock}{Demo}
			content...
		\end{exampleblock}
	\end{frame}

	\begin{frame}{Validación del sistema (II)}
		content...
	\end{frame}
	
	% ---------------------
	
	\section{Conclusiones}
	
	\begin{frame}{Conclusiones}
		\begin{itemize}
			\item asdsad
			\item asdasd
		\end{itemize}
	
		\begin{exampleblock}{Propuestas futuras}
			\begin{itemize}
				\item asd
				
				\item asd
			\end{itemize}
		\end{exampleblock}
	\end{frame}
	
	% ---------------------
	
	\section{Bibliografía}
	
	\begin{frame}{Bibliografía}
		\begin{itemize}
		    \item La contenida en la memoria del proyecto: páginas 53 - 54
		\end{itemize}
	\end{frame}
	
	\begin{frame}
	    \begin{center}
	        \vspace{30px}
	        \Large \textbf{Muchas gracias por su atención} \\
	        \vspace{30px}
	        \large ¿Preguntas? \\
	        \vspace{60px}
	        
	        Enlace a la aplicación: \\
	        \href{https://tfm-api.ranii.pro:8443/docs}{\texttt{https://tfm-api.ranii.pro:8443/}}
	    \end{center}
	\end{frame}
	
\end{document}